% are we philosophising too early?
% are we philosophising at the end?
%  add lessons at end
\documentclass[xcolor=dvipsnames,14pt]{beamer}
\beamertemplatenavigationsymbolsempty%Turn off the navbar
\usepackage{geometry}
\usepackage{minted}
\usepackage{graphicx}
\usepackage{codehilite}
\usepackage{macros}


\newcommand{\punt}[1]{}
\newcommand{\smmpunt}[1]{#1}

\begin{document}
\title{SuperMalloc}
\subtitle{A fast HTM-friendly memory allocator with a small footprint}

% Promise: By the end of this talk you will know how to make malloc 3 times faster on worst-case workloads, and you'll learn some tricks that can be applied to other performance.


\author{Bradley C. Kuszmaul \\ MIT}
\date{ACM SIGPLAN International \\ Symposium on Memory Management \\ June 14, 2015}
\frame{\titlepage}

\begin{frame}[fragile]
\frametitle{Malloc and Free}
\begin{minted}[mathescape]{c}
void* malloc(size_t s);
\end{minted}

Effect: Allocate and return a pointer to a block of memory containing at least "s" bytes.

\begin{minted}{c}
void free(void *p);
\end{minted}

Effect: "p" is a pointer to a block of memory returned by "malloc()".  Deallocate the block.
\end{frame}

\begin{frame}[fragile]
\frametitle{Aligned Allocation}

\begin{minted}[mathescape]{c}
void* memalign(size_t alignment, size_t s);
\end{minted}


Effect: Allocate and return a pointer to a block of memory containing at least "s" bytes.  
The returned pointer shall be a multiple of "alignment".  That is,
\begin{center}
"0 == (size_t)(memalign(a, s)) % a}"
\end{center}

Requires: "a" is a power of two.
\end{frame}

\newcommand{\zerobox}[1]{\parbox[b][0in][b]{0in}{#1}}

\begin{frame}
\frametitle{DLmalloc}

\hfill\zerobox{\includegraphics[width=1in]{pictures/doug_lea.png}}\hspace{1in}
%\hfill\parbox[b][0in][b]{0in}{\includegraphics[width=1in]{pictures/doug_lea.png}}\hspace{1in}

Linux libc employs Doug Lea's malloc, which dates from 1987.

\begin{itemize}
\item DLmalloc suffers high space overhead (an 8-byte boundary tag on every object, and poor space reclamation).
\item DLmalloc is slow (especially on multithreaded code, where it uses a monolithic lock).
\end{itemize}

To address these problems, allocators such as Hoard, TCmalloc, JEmalloc have appeared.

\end{frame}

\begin{frame}[fragile]
\frametitle{Modern Allocators Employ Thread Caches}

\begin{minted}[mathescape]{c}
__thread LinkedList *cache[N_SIZE_CLASSES];

void *alloc(size_t s) {
  int size_class = size_2_class(s);
  void *ret = cache[size_class];
  if (ret != NULL) {
    cache[size_class] = ret->next;
    return ret;
  }
  ...
\end{minted}

A call to the allocator requires no locking if it can be satisfied by the cache .
\end{frame}

\begin{frame}[fragile]
\frametitle{Thread Caches Are Really Fast}

\begin{minted}[mathescape]{c}
while (1) {
  free(malloc(8));
}
\end{minted}

\begin{itemize}
\item Used by the original Cilk allocator \zerobox{\hspace*{-1in}{\raisebox{.5cm}{\includegraphics[height=1in]{pictures/blumofe.jpg} \includegraphics[height=1in]{pictures/cel.jpg}}}}
[BlumofeLe94], the early STL allocator [SGI97], PTmalloc [Gloger06], LKmalloc[LarsonKr98], and TCmalloc [Google13].


\item Common case uses no locks, so it's really fast.
\item Must take a little care to avoid false cache-line sharing.
\item What goes wrong?
\end{itemize}
\end{frame}

\begin{frame}[fragile]
\frametitle{The Malloc-test Benchmark}

\begin{itemize}
\item Benchmark due to Lever and Boreham 2000.
%\textit{malloc() Performance in a Multithreaded Linux
%  Environment}, USENIX 2000.

\item $k$ allocator threads, $k$ deallocator threads.

\item Each allocator thread calls "malloc()" as fast as it can and hands the object to a deallocator thread, which calls "free()" as fast as it can.

\item It's a tough (maybe the toughest) benchmark, because per-thread caches become unbalanced.

\item The code is racy, and produces noisy data.  I sped it up by a factor of two, and de-noised it.
\end{itemize}

\end{frame}

\begin{frame}[fragile]
\frametitle{Recent Allocators Also Improved Space}

\begin{itemize}
\item Hoard [BergerMcBl00] provided the first space bound for allocators
with thread caches. 
\item TBBMalloc [KukanovVo07] adopted a later Cilk allocator (but still has a footprint problem sometimes).
\item JEmalloc [Evans06] seems to be the best incumbent.
\end{itemize}

{\tiny
\begin{tabular}{c@{}c@{}c@{}c@{}c}
\includegraphics[height=1in]{pictures/emery-berger-headshot.jpg}&
\includegraphics[height=1in]{pictures/mckinley.jpg}&
\includegraphics[height=1in]{pictures/kukanov.jpg}&
\includegraphics[height=1in]{pictures/voss.jpg}&
\includegraphics[height=1in]{pictures/evans-headshot.jpg}
\\ Berger & McKinley & Kukanov & Voss & Evans 
\end{tabular}
}
\end{frame}

\begin{frame}
\frametitle{Malloc-test On 4-core Haswell (With Hyperthreading)}

% GNUPLOT: LaTeX picture with Postscript
\begingroup
  \fontfamily{ptm}%
  \selectfont
  \makeatletter
  \providecommand\color[2][]{%
    \GenericError{(gnuplot) \space\space\space\@spaces}{%
      Package color not loaded in conjunction with
      terminal option `colourtext'%
    }{See the gnuplot documentation for explanation.%
    }{Either use 'blacktext' in gnuplot or load the package
      color.sty in LaTeX.}%
    \renewcommand\color[2][]{}%
  }%
  \providecommand\includegraphics[2][]{%
    \GenericError{(gnuplot) \space\space\space\@spaces}{%
      Package graphicx or graphics not loaded%
    }{See the gnuplot documentation for explanation.%
    }{The gnuplot epslatex terminal needs graphicx.sty or graphics.sty.}%
    \renewcommand\includegraphics[2][]{}%
  }%
  \providecommand\rotatebox[2]{#2}%
  \@ifundefined{ifGPcolor}{%
    \newif\ifGPcolor
    \GPcolortrue
  }{}%
  \@ifundefined{ifGPblacktext}{%
    \newif\ifGPblacktext
    \GPblacktextfalse
  }{}%
  % define a \g@addto@macro without @ in the name:
  \let\gplgaddtomacro\g@addto@macro
  % define empty templates for all commands taking text:
  \gdef\gplbacktext{}%
  \gdef\gplfronttext{}%
  \makeatother
  \ifGPblacktext
    % no textcolor at all
    \def\colorrgb#1{}%
    \def\colorgray#1{}%
  \else
    % gray or color?
    \ifGPcolor
      \def\colorrgb#1{\color[rgb]{#1}}%
      \def\colorgray#1{\color[gray]{#1}}%
      \expandafter\def\csname LTw\endcsname{\color{white}}%
      \expandafter\def\csname LTb\endcsname{\color{black}}%
      \expandafter\def\csname LTa\endcsname{\color{black}}%
      \expandafter\def\csname LT0\endcsname{\color[rgb]{1,0,0}}%
      \expandafter\def\csname LT1\endcsname{\color[rgb]{0,1,0}}%
      \expandafter\def\csname LT2\endcsname{\color[rgb]{0,0,1}}%
      \expandafter\def\csname LT3\endcsname{\color[rgb]{1,0,1}}%
      \expandafter\def\csname LT4\endcsname{\color[rgb]{0,1,1}}%
      \expandafter\def\csname LT5\endcsname{\color[rgb]{1,1,0}}%
      \expandafter\def\csname LT6\endcsname{\color[rgb]{0,0,0}}%
      \expandafter\def\csname LT7\endcsname{\color[rgb]{1,0.3,0}}%
      \expandafter\def\csname LT8\endcsname{\color[rgb]{0.5,0.5,0.5}}%
    \else
      % gray
      \def\colorrgb#1{\color{black}}%
      \def\colorgray#1{\color[gray]{#1}}%
      \expandafter\def\csname LTw\endcsname{\color{white}}%
      \expandafter\def\csname LTb\endcsname{\color{black}}%
      \expandafter\def\csname LTa\endcsname{\color{black}}%
      \expandafter\def\csname LT0\endcsname{\color{black}}%
      \expandafter\def\csname LT1\endcsname{\color{black}}%
      \expandafter\def\csname LT2\endcsname{\color{black}}%
      \expandafter\def\csname LT3\endcsname{\color{black}}%
      \expandafter\def\csname LT4\endcsname{\color{black}}%
      \expandafter\def\csname LT5\endcsname{\color{black}}%
      \expandafter\def\csname LT6\endcsname{\color{black}}%
      \expandafter\def\csname LT7\endcsname{\color{black}}%
      \expandafter\def\csname LT8\endcsname{\color{black}}%
    \fi
  \fi
  \setlength{\unitlength}{0.0500bp}%
  \begin{picture}(4320.00,2580.00)%
    \gplgaddtomacro\gplbacktext{%
      \csname LTb\endcsname%
      \put(538,558){\makebox(0,0)[r]{\strut{}\footnotesize 0}}%
      \csname LTb\endcsname%
      \put(538,1467){\makebox(0,0)[r]{\strut{}\footnotesize$50$M}}%
      \csname LTb\endcsname%
      \put(538,2266){\makebox(0,0)[r]{\strut{}\footnotesize $94$M}}%
      \csname LTb\endcsname%
      \put(837,298){\makebox(0,0){\strut{} 1}}%
      \csname LTb\endcsname%
      \put(1327,298){\makebox(0,0){\strut{} 2}}%
      \csname LTb\endcsname%
      \put(2308,298){\makebox(0,0){\strut{} 4}}%
      \csname LTb\endcsname%
      \put(3289,298){\makebox(0,0){\strut{} 6}}%
      \csname LTb\endcsname%
      \put(4270,298){\makebox(0,0){\strut{} 8}}%
      \csname LTb\endcsname%
      \put(139,1466){\rotatebox{-270}{\makebox(0,0){\small \texttt{malloc()}'s per second}}}%
      \csname LTb\endcsname%
      \put(2516,19){\makebox(0,0){\small Producer threads}}%
      \put(2516,2282){\makebox(0,0){\strut{}}}%
      \csname LTb\endcsname%
      \put(2516,2281){\makebox(0,0){\strut{}}}%
      \colorrgb{0.00,0.39,0.00}%
      \put(2799,2375){\makebox(0,0)[r]{\strut{}\footnotesize SuperMalloc with prefetching}}%
      \colorrgb{0.00,0.39,0.63}%
      \put(2595,2003){\makebox(0,0)[l]{\strut{}\footnotesize SuperMalloc no prefetch}}%
      \colorrgb{0.63,0.00,0.00}%
      \put(2308,722){\makebox(0,0)[r]{\strut{}\footnotesize DLmalloc}}%
      \colorrgb{0.63,0.00,0.63}%
      \put(4147,1414){\makebox(0,0)[r]{\strut{}\footnotesize Hoard}}%
      \colorrgb{0.00,0.00,0.63}%
      \put(2799,1414){\makebox(0,0){\strut{}\footnotesize JEmalloc}}%
      \colorrgb{0.00,0.63,0.63}%
      \put(2308,1285){\makebox(0,0)[r]{\strut{}\footnotesize TBBmalloc}}%
    }%
    \gplgaddtomacro\gplfronttext{%
    }%
    \gplbacktext
    \put(0,0){\includegraphics{new-malloc-test-1K-lutestring-aggregated}}%
    \gplfronttext
  \end{picture}%
\endgroup

\end{frame}
\begin{frame}
\frametitle{Malloc-test On 16-core Sandy Bridge (With Hyperthreading)}

% GNUPLOT: LaTeX picture with Postscript
\begingroup
  \fontfamily{ptm}%
  \selectfont
  \makeatletter
  \providecommand\color[2][]{%
    \GenericError{(gnuplot) \space\space\space\@spaces}{%
      Package color not loaded in conjunction with
      terminal option `colourtext'%
    }{See the gnuplot documentation for explanation.%
    }{Either use 'blacktext' in gnuplot or load the package
      color.sty in LaTeX.}%
    \renewcommand\color[2][]{}%
  }%
  \providecommand\includegraphics[2][]{%
    \GenericError{(gnuplot) \space\space\space\@spaces}{%
      Package graphicx or graphics not loaded%
    }{See the gnuplot documentation for explanation.%
    }{The gnuplot epslatex terminal needs graphicx.sty or graphics.sty.}%
    \renewcommand\includegraphics[2][]{}%
  }%
  \providecommand\rotatebox[2]{#2}%
  \@ifundefined{ifGPcolor}{%
    \newif\ifGPcolor
    \GPcolortrue
  }{}%
  \@ifundefined{ifGPblacktext}{%
    \newif\ifGPblacktext
    \GPblacktextfalse
  }{}%
  % define a \g@addto@macro without @ in the name:
  \let\gplgaddtomacro\g@addto@macro
  % define empty templates for all commands taking text:
  \gdef\gplbacktext{}%
  \gdef\gplfronttext{}%
  \makeatother
  \ifGPblacktext
    % no textcolor at all
    \def\colorrgb#1{}%
    \def\colorgray#1{}%
  \else
    % gray or color?
    \ifGPcolor
      \def\colorrgb#1{\color[rgb]{#1}}%
      \def\colorgray#1{\color[gray]{#1}}%
      \expandafter\def\csname LTw\endcsname{\color{white}}%
      \expandafter\def\csname LTb\endcsname{\color{black}}%
      \expandafter\def\csname LTa\endcsname{\color{black}}%
      \expandafter\def\csname LT0\endcsname{\color[rgb]{1,0,0}}%
      \expandafter\def\csname LT1\endcsname{\color[rgb]{0,1,0}}%
      \expandafter\def\csname LT2\endcsname{\color[rgb]{0,0,1}}%
      \expandafter\def\csname LT3\endcsname{\color[rgb]{1,0,1}}%
      \expandafter\def\csname LT4\endcsname{\color[rgb]{0,1,1}}%
      \expandafter\def\csname LT5\endcsname{\color[rgb]{1,1,0}}%
      \expandafter\def\csname LT6\endcsname{\color[rgb]{0,0,0}}%
      \expandafter\def\csname LT7\endcsname{\color[rgb]{1,0.3,0}}%
      \expandafter\def\csname LT8\endcsname{\color[rgb]{0.5,0.5,0.5}}%
    \else
      % gray
      \def\colorrgb#1{\color{black}}%
      \def\colorgray#1{\color[gray]{#1}}%
      \expandafter\def\csname LTw\endcsname{\color{white}}%
      \expandafter\def\csname LTb\endcsname{\color{black}}%
      \expandafter\def\csname LTa\endcsname{\color{black}}%
      \expandafter\def\csname LT0\endcsname{\color{black}}%
      \expandafter\def\csname LT1\endcsname{\color{black}}%
      \expandafter\def\csname LT2\endcsname{\color{black}}%
      \expandafter\def\csname LT3\endcsname{\color{black}}%
      \expandafter\def\csname LT4\endcsname{\color{black}}%
      \expandafter\def\csname LT5\endcsname{\color{black}}%
      \expandafter\def\csname LT6\endcsname{\color{black}}%
      \expandafter\def\csname LT7\endcsname{\color{black}}%
      \expandafter\def\csname LT8\endcsname{\color{black}}%
    \fi
  \fi
  \setlength{\unitlength}{0.0500bp}%
  \begin{picture}(5760.00,3240.00)%
    \gplgaddtomacro\gplbacktext{%
      \csname LTb\endcsname%
      \put(640,558){\makebox(0,0)[r]{\strut{}\footnotesize 0}}%
      \csname LTb\endcsname%
      \put(640,1482){\makebox(0,0)[r]{\strut{}\footnotesize$50$M}}%
      \csname LTb\endcsname%
      \put(640,2407){\makebox(0,0)[r]{\strut{}\footnotesize$100$M}}%
      \csname LTb\endcsname%
      \put(640,2998){\makebox(0,0)[r]{\strut{}\footnotesize $132$M}}%
      \csname LTb\endcsname%
      \put(969,298){\makebox(0,0){\strut{} 1}}%
      \csname LTb\endcsname%
      \put(2036,298){\makebox(0,0){\strut{} 8}}%
      \csname LTb\endcsname%
      \put(3257,298){\makebox(0,0){\strut{} 16}}%
      \csname LTb\endcsname%
      \put(4477,298){\makebox(0,0){\strut{} 24}}%
      \csname LTb\endcsname%
      \put(5698,298){\makebox(0,0){\strut{} 32}}%
      \csname LTb\endcsname%
      \put(37,1796){\rotatebox{-270}{\makebox(0,0){\small \texttt{malloc()}'s per second}}}%
      \csname LTb\endcsname%
      \put(3287,19){\makebox(0,0){\small Producer threads}}%
      \put(3287,2942){\makebox(0,0){\strut{}}}%
      \csname LTb\endcsname%
      \put(3287,2941){\makebox(0,0){\strut{}}}%
      \colorrgb{0.00,0.39,0.00}%
      \put(3104,2776){\makebox(0,0)[r]{\strut{}\footnotesize SuperMalloc}}%
      \colorrgb{0.63,0.00,0.00}%
      \put(3562,669){\makebox(0,0)[r]{\strut{}\footnotesize DLmalloc}}%
      \colorrgb{0.63,0.00,0.63}%
      \put(5698,1575){\makebox(0,0)[r]{\strut{}\footnotesize Hoard}}%
      \colorrgb{0.00,0.00,0.63}%
      \put(3867,1575){\makebox(0,0){\strut{}\footnotesize JEmalloc}}%
      \colorrgb{0.00,0.63,0.63}%
      \put(3104,1390){\makebox(0,0)[r]{\strut{}\footnotesize TBBmalloc}}%
    }%
    \gplgaddtomacro\gplfronttext{%
    }%
    \gplbacktext
    \put(0,0){\includegraphics{new-malloc-test-1K-tempo-aggregated}}%
    \gplfronttext
  \end{picture}%
\endgroup

\end{frame}

\begin{frame}
\frametitle{Explaining the Performance}

\begin{itemize}
\item How DLmalloc works.
\item How Hoard \& JEmalloc work.
\item How SuperMalloc works.
\end{itemize}

\end{frame}

\begin{frame}
\frametitle{DLmalloc Employs Bins}
A bin is a doubly linked list of free objects that are all close to the same size.

\vfill

For example, DLmalloc employs 32 ``small'' bins of size 8, 16, 24, 32,
\ldots, 256 bytes respectively.  If you allocate 12 bytes, you get a 16-byte object.

\end{frame}

\begin{frame}[fragile]
\frametitle{DLmalloc Employs Boundary Tags}

A 64-bit integer encoding "{size, in_use, prev_in_use}" precedes each object.

Also stores "size" at the end of free objects.

\vspace*{-0.5cm}
\begin{center}
\begin{tabular}{@{}c@{}c}
\begin{tabular}{c}
\vspace*{0.5in} \\
\includegraphics[width=1in]{pictures/knuth.jpg}  \\
\tiny Boundary tags \\
\tiny due to Knuth.
\end{tabular}
&
\begin{tabular}{|l|l|}
allocated                         & $\ldots$ user data $\ldots$ \\ \hline
allocated                         & "{32, true, true}"          \\ \cline{2-2}
\multicolumn{1}{|r|}{(32 bytes)}  & $\ldots$ user data $\ldots$ \\ \hline
free                              & "{48, false, true}"         \\ \cline{2-2}
\multicolumn{1}{|r|}{(48 bytes)}  & "void* prev_in_bin;" \\ \cline{2-2}
                                  & "void* next_in_bin;" \\ \cline{2-2}
                                  & $\ldots$ unused space $\ldots$ \\ \cline{2-2}
                                  & "48"                 \\ \hline
allocated                         & "96, true, false"    \\ \cline{2-2}
\multicolumn{1}{|r|}{(96 bytes)}  & $\ldots$ user data $\ldots$ \\ \hline
                &                                   \\
\end{tabular}
\end{tabular}
\end{center}
\end{frame}

\begin{frame}
\frametitle{DLmalloc Is Simple But Slow}

\begin{tabular}{rrr}
         & lines of code & malloc\_test speed \\
DLmalloc &    6,281 &  3.7M/s \\
Hoard    &   16,948 &  17M/s \\
JEmalloc    & 22,230 & 38M/s\\
SuperMalloc & 3,571 & 131M/s \\
\end{tabular}

\end{frame}

\begin{frame}
\frametitle{Running Faster Than DLmalloc}

\begin{itemize}
\item DLmalloc uses a monolithic lock.
\item Hoard employs per-thread caches to reduce lock-acquisition frequency, along with a provable space bound.
\item JEmalloc uses many of the same tricks, provides a weaker space bound in theory, but seems better in practice.
\item Each thread keeps a cache of allocated objects.
\item Each thread has an ``arena''.  When the thread cache is empty, allocate out of the arena, using a per-arena lock.
\end{itemize}

\end{frame}

\begin{frame}[fragile]
\frametitle{Running Smaller Than DLmalloc}

\begin{itemize}
\item Allocate 4MiB \textit{chunks} using "mmap()".
\item Objects within a chunk are all the same size (to first order).
\item Only $1$-bit/object overhead.
\item A red-black tree indexed on the chunk number, "p >> 22", provides a chunk's object size.
\item In an arena, JEmalloc allocates the object with the least address,
tending to depopulate pages.
\item Hoard is similar, except that it allocates from the fullest page.
\end{itemize}
\end{frame}

\begin{frame}
\frametitle{Under the Hood of SuperMalloc}

SuperMalloc is fast and small for many reasons.  Two main ideas:
\begin{itemize}
\item Virtual-memory tricks.
\item Reduced cache cost.
\end{itemize}
\end{frame}

\begin{frame}
\frametitle{Virtual Memory}
\end{frame}


\begin{frame}[fragile]
\frametitle{x86-64 Address Space}

\begin{picture}(0,0)%
\includegraphics{x86-addresses_pdftex.pdf}%
\end{picture}%
\setlength{\unitlength}{2486sp}%
%
\begingroup\makeatletter\ifx\SetFigFont\undefined%
\gdef\SetFigFont#1#2#3#4#5{%
  \reset@font\fontsize{#1}{#2pt}%
  \fontfamily{#3}\fontseries{#4}\fontshape{#5}%
  \selectfont}%
\fi\endgroup%
\begin{picture}(7372,5650)(1311,-5597)
\put(4726,-961){\makebox(0,0)[rb]{\smash{{\SetFigFont{14}{16.8}{\rmdefault}{\mddefault}{\updefault}{\color[rgb]{0,0,0}x86-64 address}%
}}}}
\put(4726,-1471){\makebox(0,0)[rb]{\smash{{\SetFigFont{14}{16.8}{\rmdefault}{\mddefault}{\updefault}{\color[rgb]{0,0,0}space is 64 bits.}%
}}}}
\put(4726,-1981){\makebox(0,0)[rb]{\smash{{\SetFigFont{14}{16.8}{\rmdefault}{\mddefault}{\updefault}{\color[rgb]{0,0,0}Only 48 bits work.}%
}}}}
\put(4726,-2491){\makebox(0,0)[rb]{\smash{{\SetFigFont{14}{16.8}{\rmdefault}{\mddefault}{\updefault}{\color[rgb]{0,0,0}Viewed as signed}%
}}}}
\put(4726,-3001){\makebox(0,0)[rb]{\smash{{\SetFigFont{14}{16.8}{\rmdefault}{\mddefault}{\updefault}{\color[rgb]{0,0,0}64-bit numbers,}%
}}}}
\put(4726,-3511){\makebox(0,0)[rb]{\smash{{\SetFigFont{14}{16.8}{\rmdefault}{\mddefault}{\updefault}{\color[rgb]{0,0,0}valid addresses are}%
}}}}
\put(4726,-4021){\makebox(0,0)[rb]{\smash{{\SetFigFont{14}{16.8}{\rmdefault}{\mddefault}{\updefault}{\color[rgb]{0,0,0}$-(2^{47})$--$(2^{47}-1)$}%
}}}}
\put(8551,-1411){\rotatebox{90.0}{\makebox(0,0)[rb]{\smash{{\SetFigFont{14}{16.8}{\rmdefault}{\mddefault}{\updefault}{\color[rgb]{0,0,0}(not to scale)}%
}}}}}
\put(6526,-3886){\makebox(0,0)[rb]{\smash{{\SetFigFont{14}{16.8}{\rmdefault}{\mddefault}{\updefault}{\color[rgb]{0,0,0}$-2^{47}$}%
}}}}
\put(6526,-2761){\makebox(0,0)[rb]{\smash{{\SetFigFont{14}{16.8}{\rmdefault}{\mddefault}{\updefault}{\color[rgb]{0,0,0}$0$}%
}}}}
\put(6526,-1636){\makebox(0,0)[rb]{\smash{{\SetFigFont{14}{16.8}{\rmdefault}{\mddefault}{\updefault}{\color[rgb]{0,0,0}$2^{47}$}%
}}}}
\put(6526,-286){\makebox(0,0)[rb]{\smash{{\SetFigFont{14}{16.8}{\rmdefault}{\mddefault}{\updefault}{\color[rgb]{0,0,0}$2^{63}$}%
}}}}
\put(6526,-5461){\makebox(0,0)[rb]{\smash{{\SetFigFont{14}{16.8}{\rmdefault}{\mddefault}{\updefault}{\color[rgb]{0,0,0}$-2^{63}$}%
}}}}
\end{picture}%

\end{frame}

\begin{frame}[fragile]
\frametitle{Uncommitted Memory}

\begin{itemize}
\item The kernel allocates physical memory lazily.
\item When you map memory, virtual addresses are allocated, but the page table is set up to point to a special zero-page.
\item When you write to a previously unwritten page, the kernel allocates memory, \textit{committing} physical memory to the virtual page.
\item You can decommit physical memory, returning it to the kernel, via "madvise()".
\end{itemize}

\end{frame}

\begin{frame}[fragile]
\frametitle{Which Object Is Being Freed?}

On "free(p)", we need to know which bitmap bit to clear.
\begin{itemize}
\item All objects in the same chunks are the same size.  SuperMalloc Chunks are 2MiB.

\item Compute:

  "chunk  = (p >> 21) & ((1<<27)-1);"

  "bitnum = (p&((1<<21)-1)) / objsize;"

\item How to determine "objsize"?
JEmalloc and Hoard employ a red-black tree.
\end{itemize}
\end{frame}

\punt{
\begin{frame}[fragile]
\frametitle{Use An Array, Not A Tree}

SuperMalloc employs an array.
\begin{itemize}
\item There are only $2^{27}$ possible chunks.  Need 32 bits to code the size, a total of
$2^{29}$ bytes.

\item "mmap()" mostly returns adjacent blocks.

\item E.g., a program that uses $128$ GiB of allocated data needs only
  $65,536$ chunks.  The table consumes 512MiB of address space, but
  only a 256KiB of RSS.

\item Determining an object's size requires $O(1)$ instructions, cache misses, and TLB misses.
\end{itemize}
\end{frame}
}

\begin{frame}[fragile]
\frametitle{Use An Array, Not A Tree}
\hspace*{-0.5cm}\input{chunk-array_pdftex}
\end{frame}

\newcommand{\pw}[2]{\begin{minipage}{#1}\raggedright #2\end{minipage}}

\smmpunt{
\begin{frame}
\frametitle{Fullest-Page Heap}
\hspace*{-0.5cm}\begin{picture}(0,0)%
\includegraphics{small-number-heap_pdftex.pdf}%
\end{picture}%
\setlength{\unitlength}{3315sp}%
%
\begingroup\makeatletter\ifx\SetFigFont\undefined%
\gdef\SetFigFont#1#2#3#4#5{%
  \reset@font\fontsize{#1}{#2pt}%
  \fontfamily{#3}\fontseries{#4}\fontshape{#5}%
  \selectfont}%
\fi\endgroup%
\begin{picture}(6270,4188)(4036,-4090)
\put(9226,-781){\makebox(0,0)[lb]{\smash{{\SetFigFont{10}{12.0}{\rmdefault}{\mddefault}{\updefault}{\color[rgb]{0,0,0}full pages}%
}}}}
\put(4096,-1546){\makebox(0,0)[lb]{\smash{{\SetFigFont{10}{12.0}{\rmdefault}{\mddefault}{\updefault}{\color[rgb]{0,0,0}Could use a heap.}%
}}}}
\put(4096,-1771){\makebox(0,0)[lb]{\smash{{\SetFigFont{10}{12.0}{\rmdefault}{\mddefault}{\updefault}{\color[rgb]{0,0,0}But all keys are}%
}}}}
\put(4096,-1996){\makebox(0,0)[lb]{\smash{{\SetFigFont{10}{12.0}{\rmdefault}{\mddefault}{\updefault}{\color[rgb]{0,0,0}small integers.}%
}}}}
\put(8461,-2176){\makebox(0,0)[lb]{\smash{{\SetFigFont{10}{12.0}{\rmdefault}{\mddefault}{\updefault}{\color[rgb]{0,0,0}pages with 4 free slots}%
}}}}
\put(8326,-3436){\makebox(0,0)[lb]{\smash{{\SetFigFont{10}{12.0}{\rmdefault}{\mddefault}{\updefault}{\color[rgb]{0,0,0}pages with 6 free slots}%
}}}}
\put(4096,-61){\makebox(0,0)[lb]{\smash{{\SetFigFont{10}{12.0}{\rmdefault}{\mddefault}{\updefault}{\color[rgb]{0,0,0}To allocate an object, we want to allocate from the fullest possible page.}%
}}}}
\put(5266,-736){\makebox(0,0)[rb]{\smash{{\SetFigFont{10}{12.0}{\rmdefault}{\mddefault}{\updefault}{\color[rgb]{0,0,0}fullest index:}%
}}}}
\put(4051,-2671){\makebox(0,0)[lb]{\smash{{\SetFigFont{10}{12.0}{\rmdefault}{\mddefault}{\updefault}{\color[rgb]{0,0,0}To insert, add to list and}%
}}}}
\put(4051,-3346){\makebox(0,0)[lb]{\smash{{\SetFigFont{10}{12.0}{\rmdefault}{\mddefault}{\updefault}{\color[rgb]{0,0,0}To remove may require a}%
}}}}
\put(4051,-2896){\makebox(0,0)[lb]{\smash{{\SetFigFont{10}{12.0}{\rmdefault}{\mddefault}{\updefault}{\color[rgb]{0,0,0}update fullest index, in}%
}}}}
\put(4051,-3121){\makebox(0,0)[lb]{\smash{{\SetFigFont{10}{12.0}{\rmdefault}{\mddefault}{\updefault}{\color[rgb]{0,0,0}$O(1)$ time.}%
}}}}
\put(4051,-3571){\makebox(0,0)[lb]{\smash{{\SetFigFont{10}{12.0}{\rmdefault}{\mddefault}{\updefault}{\color[rgb]{0,0,0}search to update fullest}%
}}}}
\put(4051,-3796){\makebox(0,0)[lb]{\smash{{\SetFigFont{10}{12.0}{\rmdefault}{\mddefault}{\updefault}{\color[rgb]{0,0,0}index in $O(1)$ amortized time.}%
}}}}
\put(4051,-4021){\makebox(0,0)[lb]{\smash{{\SetFigFont{10}{12.0}{\rmdefault}{\mddefault}{\updefault}{\color[rgb]{0,0,0}Michael Bender says I should use a bitmap instead of fullest index.}%
}}}}
\end{picture}%

\end{frame}

\begin{frame}
\frametitle{Synchronization}
\end{frame}
}

\begin{frame}[fragile]
\frametitle{Costs of Locking vs. Cache Contention}

I measured the cost of updating a variable in a multithreaded code.

\begin{tabular}{rr}
                                                  global variable &    193.6ns \\
            per cpu (always call "sched_getcpu()"  &     30.2ns \\
per cpu (cache getcpu, refresh/32) &     17.0ns \\
                                                      per thread &      3.1ns \\
                                                  local in stack &      3.1ns \\
\end{tabular}

How to use this information:  Make a per-thread cache that's just big enough to amortize the locking instruction, and then use a per-CPU cache.
\end{frame}

\begin{frame}
\frametitle{SuperMalloc Cache Architecture}
\begin{picture}(0,0)%
\includegraphics{cache-architecture_pdftex.pdf}%
\end{picture}%
\setlength{\unitlength}{2901sp}%
%
\begingroup\makeatletter\ifx\SetFigFont\undefined%
\gdef\SetFigFont#1#2#3#4#5{%
  \reset@font\fontsize{#1}{#2pt}%
  \fontfamily{#3}\fontseries{#4}\fontshape{#5}%
  \selectfont}%
\fi\endgroup%
\begin{picture}(6678,4524)(799,-4753)
\put(5401,-2536){\makebox(0,0)[lb]{\smash{{\SetFigFont{8}{9.6}{\rmdefault}{\mddefault}{\updefault}{\color[rgb]{0,0,0}Move 10 objects in $O(1)$ time.}%
}}}}
\put(5401,-2311){\makebox(0,0)[lb]{\smash{{\SetFigFont{8}{9.6}{\rmdefault}{\mddefault}{\updefault}{\color[rgb]{0,0,0}Needs Mutual Exclusion}%
}}}}
\put(2836,-4381){\makebox(0,0)[b]{\smash{{\SetFigFont{8}{9.6}{\rmdefault}{\mddefault}{\updefault}{\color[rgb]{0,0,0}Global Data}%
}}}}
\put(2836,-4606){\makebox(0,0)[b]{\smash{{\SetFigFont{8}{9.6}{\rmdefault}{\mddefault}{\updefault}{\color[rgb]{0,0,0}Structure}%
}}}}
\put(2836,-3436){\makebox(0,0)[b]{\smash{{\SetFigFont{8}{9.6}{\rmdefault}{\mddefault}{\updefault}{\color[rgb]{0,0,0}Global Cache}%
}}}}
\put(2836,-3661){\makebox(0,0)[b]{\smash{{\SetFigFont{8}{9.6}{\rmdefault}{\mddefault}{\updefault}{\color[rgb]{0,0,0}$P$ MiB}%
}}}}
\put(1801,-2311){\makebox(0,0)[b]{\smash{{\SetFigFont{8}{9.6}{\rmdefault}{\mddefault}{\updefault}{\color[rgb]{0,0,0}CPU Cache}%
}}}}
\put(1801,-2536){\makebox(0,0)[b]{\smash{{\SetFigFont{8}{9.6}{\rmdefault}{\mddefault}{\updefault}{\color[rgb]{0,0,0}(1 MiB per size)}%
}}}}
\put(4186,-2311){\makebox(0,0)[b]{\smash{{\SetFigFont{8}{9.6}{\rmdefault}{\mddefault}{\updefault}{\color[rgb]{0,0,0}CPU Cache}%
}}}}
\put(4186,-2536){\makebox(0,0)[b]{\smash{{\SetFigFont{8}{9.6}{\rmdefault}{\mddefault}{\updefault}{\color[rgb]{0,0,0}(1 MiB per size)}%
}}}}
\put(6211,-421){\makebox(0,0)[b]{\smash{{\SetFigFont{8}{9.6}{\rmdefault}{\mddefault}{\updefault}{\color[rgb]{0,0,0}Thread}%
}}}}
\put(6166,-1186){\makebox(0,0)[b]{\smash{{\SetFigFont{8}{9.6}{\rmdefault}{\mddefault}{\updefault}{\color[rgb]{0,0,0}Thread Cache}%
}}}}
\put(6166,-1411){\makebox(0,0)[b]{\smash{{\SetFigFont{8}{9.6}{\rmdefault}{\mddefault}{\updefault}{\color[rgb]{0,0,0}(10 per size)}%
}}}}
\put(4636,-421){\makebox(0,0)[b]{\smash{{\SetFigFont{8}{9.6}{\rmdefault}{\mddefault}{\updefault}{\color[rgb]{0,0,0}Thread}%
}}}}
\put(4591,-1186){\makebox(0,0)[b]{\smash{{\SetFigFont{8}{9.6}{\rmdefault}{\mddefault}{\updefault}{\color[rgb]{0,0,0}Thread Cache}%
}}}}
\put(4591,-1411){\makebox(0,0)[b]{\smash{{\SetFigFont{8}{9.6}{\rmdefault}{\mddefault}{\updefault}{\color[rgb]{0,0,0}(10 per size)}%
}}}}
\put(3106,-421){\makebox(0,0)[b]{\smash{{\SetFigFont{8}{9.6}{\rmdefault}{\mddefault}{\updefault}{\color[rgb]{0,0,0}Thread}%
}}}}
\put(3061,-1186){\makebox(0,0)[b]{\smash{{\SetFigFont{8}{9.6}{\rmdefault}{\mddefault}{\updefault}{\color[rgb]{0,0,0}Thread Cache}%
}}}}
\put(3061,-1411){\makebox(0,0)[b]{\smash{{\SetFigFont{8}{9.6}{\rmdefault}{\mddefault}{\updefault}{\color[rgb]{0,0,0}(10 per size)}%
}}}}
\put(1531,-421){\makebox(0,0)[b]{\smash{{\SetFigFont{8}{9.6}{\rmdefault}{\mddefault}{\updefault}{\color[rgb]{0,0,0}Thread}%
}}}}
\put(1486,-1186){\makebox(0,0)[b]{\smash{{\SetFigFont{8}{9.6}{\rmdefault}{\mddefault}{\updefault}{\color[rgb]{0,0,0}Thread Cache}%
}}}}
\put(1486,-1411){\makebox(0,0)[b]{\smash{{\SetFigFont{8}{9.6}{\rmdefault}{\mddefault}{\updefault}{\color[rgb]{0,0,0}(10 per size)}%
}}}}
\end{picture}%

\end{frame}

\begin{frame}
\frametitle{To Allocate}


\begin{enumerate}
\item Determine which \textit{size bin} to use.
\item Look in the per-thread and per-CPU caches.
\item Else, Atomically 
  \begin{enumerate}
  \item Find the fullest page in that bin with a free slot.
  \item Find a free slot in the bitmap for the page.
  \item Set the bit.
  \item Change the page's position in the fullest-page heap.
  \end{enumerate}
\item Else (nothing was found), allocate a new chunk.
\end{enumerate}
\end{frame}

\begin{frame}
\frametitle{To Free}
\begin{enumerate}
\item If the thread cache and cpu cache are full, 

 Atomically
 \begin{enumerate}
 \item Clear the bit in the free map.
 \item Change the page's position in the fullest-page heap.
 \end{enumerate}
\item Otherwise 
 \begin{enumerate}
 \item Insert the object to the thread cache.
 \item If the thread cache is full, move several objects to the per-cpu cache (in $O(1)$ time).
 \end{enumerate}
\end{enumerate}
\end{frame}

\smmpunt{
\begin{frame}
\frametitle{More Caching Issues}
\end{frame}
}

\smmpunt{}
\begin{frame}[fragile]
\frametitle{Size Bins Introduce Associativity Conflicts}
\begin{picture}(0,0)%
\includegraphics{associativity-conflicts_pdftex.pdf}%
\end{picture}%
\setlength{\unitlength}{3315sp}%
%
\begingroup\makeatletter\ifx\SetFigFont\undefined%
\gdef\SetFigFont#1#2#3#4#5{%
  \reset@font\fontsize{#1}{#2pt}%
  \fontfamily{#3}\fontseries{#4}\fontshape{#5}%
  \selectfont}%
\fi\endgroup%
\begin{picture}(5842,4432)(2935,-4077)
\put(6301,-961){\makebox(0,0)[lb]{\smash{{\SetFigFont{10}{12.0}{\rmdefault}{\mddefault}{\updefault}{\color[rgb]{0,0,0}aligned on 1024-bytes.}%
}}}}
\put(4051,-286){\makebox(0,0)[lb]{\smash{{\SetFigFont{10}{12.0}{\rmdefault}{\mddefault}{\updefault}{\color[rgb]{0,0,0}object 0}%
}}}}
\put(4051,-1186){\makebox(0,0)[lb]{\smash{{\SetFigFont{10}{12.0}{\rmdefault}{\mddefault}{\updefault}{\color[rgb]{0,0,0}object 1}%
}}}}
\put(4051,-2086){\makebox(0,0)[lb]{\smash{{\SetFigFont{10}{12.0}{\rmdefault}{\mddefault}{\updefault}{\color[rgb]{0,0,0}object 2}%
}}}}
\put(4051,-3661){\makebox(0,0)[lb]{\smash{{\SetFigFont{10}{12.0}{\rmdefault}{\mddefault}{\updefault}{\color[rgb]{0,0,0}object 60}%
}}}}
\put(6301,-2851){\makebox(0,0)[lb]{\smash{{\SetFigFont{10}{12.0}{\rmdefault}{\mddefault}{\updefault}{\color[rgb]{0,0,0}in a hardware transaction.}%
}}}}
\put(6301,-2626){\makebox(0,0)[lb]{\smash{{\SetFigFont{10}{12.0}{\rmdefault}{\mddefault}{\updefault}{\color[rgb]{0,0,0}Also cannot traverse the list }%
}}}}
\put(6301,-2086){\makebox(0,0)[lb]{\smash{{\SetFigFont{10}{12.0}{\rmdefault}{\mddefault}{\updefault}{\color[rgb]{0,0,0}without cache misses.}%
}}}}
\put(6301,-1861){\makebox(0,0)[lb]{\smash{{\SetFigFont{10}{12.0}{\rmdefault}{\mddefault}{\updefault}{\color[rgb]{0,0,0}So we cannot traverse this list}%
}}}}
\put(6301,-1636){\makebox(0,0)[lb]{\smash{{\SetFigFont{10}{12.0}{\rmdefault}{\mddefault}{\updefault}{\color[rgb]{0,0,0}Cache is 8-way associative.}%
}}}}
\put(6301,-1411){\makebox(0,0)[lb]{\smash{{\SetFigFont{10}{12.0}{\rmdefault}{\mddefault}{\updefault}{\color[rgb]{0,0,0}four cache sets.}%
}}}}
\put(6301,-1186){\makebox(0,0)[lb]{\smash{{\SetFigFont{10}{12.0}{\rmdefault}{\mddefault}{\updefault}{\color[rgb]{0,0,0}So they all end up in one of}%
}}}}
\put(6301,-736){\makebox(0,0)[lb]{\smash{{\SetFigFont{10}{12.0}{\rmdefault}{\mddefault}{\updefault}{\color[rgb]{0,0,0}The ``next'' fields all end up}%
}}}}
\put(6301,-511){\makebox(0,0)[lb]{\smash{{\SetFigFont{10}{12.0}{\rmdefault}{\mddefault}{\updefault}{\color[rgb]{0,0,0}Ends up in 1024-byte bin.}%
}}}}
\put(6301,-286){\makebox(0,0)[lb]{\smash{{\SetFigFont{10}{12.0}{\rmdefault}{\mddefault}{\updefault}{\color[rgb]{0,0,0}990-byte object}%
}}}}
\end{picture}%

\end{frame}
{}

\smmpunt{
\begin{frame}
\frametitle{Odd-sized Bins}

\begin{itemize}
\item Solution (due to Dave Dice).  Make object sizes be a prime number of cache lines.

\url{https://blogs.oracle.com/dave/entry/malloc_for_haswell_hardware_transactional}

\item In principle, odd-sized should be good enough.  Prime numbers avoid inter-bin associativity conflicts, however.

\hfill\zerobox{\raisebox{-0.6in}{\includegraphics[height=1in]{pictures/dave_dice.jpg}}}\hspace*{1in}

\end{itemize}

\end{frame}
}

\smmpunt{
\begin{frame}
\frametitle{Odd-sized Bins Performance Issues}

\begin{itemize}
\item Calculating bin number from size.
\item Calculating bitmap index from a pointer.
\end{itemize}
\end{frame}
}

\smmpunt{
\begin{frame}
\frametitle{SuperMalloc Bin Sizes}

Four size categories: small, medium, large, huge.
\end{frame}
}

\smmpunt{
\begin{frame}
\frametitle{Small Size}

\begin{itemize}
\item Small sizes are of the form $\frac{2^k}{1}$, $\frac{5 \cdot 2^k}{4}$, $\frac{3 \cdot 2^k}{2}$, $\frac{7 \cdot 2^k}{4}$.  
\item The numbers are 8, 10, 12, 14, 16, 20, 24, 28, 32, 40, 48, 56, 64, 80, 96, 112, 128, 160, 192, 224, 256, and 320.
\item These admit fast calculation of bin numbers via bit hacks.
\item Technically, some of these are bad alignments:  DLmalloc provides 8-byte aligned data, and programmers may expect it.  Probably need to remove 10, 12, 14, 20, and 28.
\end{itemize}
\end{frame}
}

\smmpunt{
\begin{frame}
\frametitle{Medium Size}
\begin{itemize}
\item Prime numbers of cache lines: 5, 7, 11, 13, 17, 23,
  $\ldots$ cache lines.

\item Throw in 9 and 15 to avoid fragmentation.

\item There are 45 small- and medium-sized bins.
\item Aligned allocations have their own bins.
\end{itemize}
\end{frame}
}

\begin{frame}
\frametitle{Prime-Sized Objects Cause Fragmentation}
\hspace*{-0.5cm}\input{folio-problem_pdftex}
\end{frame}

\begin{frame}
\frametitle{Folios}
\begin{itemize}
\item Use somewhat larger pages, called ``folios'', which contain 64 objects.
\item For example, $13$ pages hold exactly $64$ $13$-line objects.
\item No fragmentation at the end of the folio.
\item There are a few pages at the end of a chunk that are unused, but we keep those pages uncommitted.
\item For example, $39$ $13$-page folios fit in a $2$MiB chunk, with $5$ pages left over.
\end{itemize}

\end{frame}

\punt{
\begin{frame}
\frametitle{Folios}

\begin{itemize}
\item Prime-sized objects leave unused space at the end of a page.

\item Consider the 832-byte bin (13 cache lines). 

\item Four of those objects fit in a page.
\item Wastes 768 bytes at the end.
\item Use somewhat larger pages, we call ``folios'' which contain 64 objects.
\item No fragmentation at the end of the folio.
\item There are a few pages at the end that are unused, but we keep those pages uncommitted.
\end{itemize}
\end{frame}
}

\smmpunt{}
\begin{frame}[fragile]
\frametitle{Large Size}
\begin{itemize}
\item Page allocated plus a random offset.   
\item The chunk is organized by powers of two.  There is a chunk for 4-page objects, a chunk for 8-page objects, a chunk for 16-page objects, and so forth.
\item Make sure the unused pages are uncommitted (that is, not in physical memory).
\item Wastes at most $2\times$ virtual memory.
\item The chunk array tracks the number of pages.
\end{itemize}
\end{frame}
{}

\smmpunt{}
\begin{frame}[fragile]
\frametitle{Huge Size}

\begin{itemize}
\item Allocated via mmap plus a random offet. 
\item Huge objects are allocated on power-of-two chunks.  (Keep the unused part uncommitted).
\item By using power-of-two chunks we have relatively few allocation bins.
\item The chunk array tracks the number of pages.
\end{itemize}
\end{frame}
{}

\smmpunt{}
\begin{frame}[fragile]
\frametitle{Calculating Bitmap Indexes}

Given a pointer $p$:

\begin{tabular}{ll}
 Chunk number is & "C = p / chunksize". \\
 Bin is          & "B = chunkinfos[C]".  \\
 Folio size is   & "FS = foliosizes[B]". \\
 Folio number is & "FN = (p%chunksize)/FS". \\
 Object size is  & "OS = objectsizes[B]". \\
 Object number is& "ON = ((p%chunksize)%FS)/OS".
\end{tabular}

Division by "chunksize", a constant power of two, is
easy.  Division by "FS" and "OS" could be
slow.

\end{frame}
{}

\smmpunt{}
\begin{frame}[fragile]
\frametitle{Division via Multiplication-and-Shift}

Division is slow.  Multiplication is fast.

It turns out that for 32-bit values of "x",
\begin{center}
"x / 320 == (x*6871947674lu)>>41"
\end{center}

[MagenheimerPePe87] shows how to calculate these magic numbers. 

\hfill\zerobox{\raisebox{-0.6in}{\includegraphics[height=1in]{pictures/magenheimer.jpg}}}\hspace*{1in}
\end{frame}
{}

\smmpunt{}
\begin{frame}[fragile]
\frametitle{Magic Numbers for Division}

Problem: divide by "D" using multiply and shift.

Idea:  For real numbers, $\frac{x (2^{32}/D)}{2^{32}} = \frac{x}{D}$.

For integer arithmetic use:
\begin{eqnarray*}
S & = & 32+\lceil \log_2 D \rceil \\
M & = & (D-1+2^S)/D 
\end{eqnarray*}
in which case for $x<2^{32}$,
\[
\lfloor x/D \rfloor = \lfloor (xM)/2^S \rfloor.
\]

A metaprogram computes all the sizes and the magic constants.
\end{frame}
{}

\smmpunt{}
\begin{frame}[fragile]
\frametitle{Using Hardware Transactional Memory}

\begin{minted}[mathescape]{c}
while (1) {
  if (_xbegin() == _XBEGIN_STARTED) {
    if (lock) _xabort(); // subscribe to lock
    critical_section();
    _xend();
    break;
  }
  if (!try_again()) {
    acquire_lock(&lock);
    critical_section();
    release_lock(&lock);
    break;
} }
\end{minted}
\end{frame}
{}

\smmpunt{
\begin{frame}
\frametitle{Why does an HTM transaction fail?}
\begin{itemize}
\item Interrupts (such as time-slice).
\item Cache capacity (read set must fit in L2.  Write set must fit in L1).
\item Actual conflicts (two concurrent transactions have a race).
\item Conflicts with the fallback code.
\item Other random failures.  The HTM failure codes seem useless.
\end{itemize}

\end{frame}
}

\smmpunt{}
\begin{frame}[fragile]
\frametitle{Improving the Odds}

\begin{minted}[mathescape]{c}
while (1) {
  // prewait for the lock.
  while (lock) _mm_pause(); // save power.
  // prefetch needed data:
  predo_critical_section()
  if (_xbegin() == _XBEGIN_STARTED) {
    critical_section();
    if (lock) _xabort(); // late subscription.
    _xend();
    break;
  }
  if (!try_again()) {
    acquire_lock(&lock);
    critical_section();
    release_lock(&lock);
    break;
  }
}
\end{minted}
\end{frame}
{}

\smmpunt{
\begin{frame}
\frametitle{What Works?}

\begin{itemize}
\item HTM wins over spin locks for high thread counts.  Maybe need better locks?
\item Prewaiting for the lock is a big win.
\item Preloading the cache is little win.
\item Late lock subscription does not help much, and is dangerous.  

The transaction may be running between two arbitrary instructions of
the critical section protected by the lock.  How do you know that
we'll successfully get the abort?

I got rid of late lock subscription.
\item Lock fallback needed less than 0.03\% of the time.
\end{itemize}
\end{frame}
}

\begin{frame}
\frametitle{Summary of SuperMalloc Strategies}

\begin{itemize}
\item Can waste some virtual space: it's $2^{48}$ bytes.
\item Don't waste RSS ($2^{40}$ bytes on big machines).
\item Contention costs dearly, not locking;  use a per-CPU cache.
\item Make per-CPU cache smaller than L3 cache, since the application
  has cache misses anyway.
\item Thread cache just big enough to reduce locking overhead: about 10 objects.
\item Use hardware transactional memory (HTM).
\item Use simple data structures.  E.g., Use a big array instead of a red-black tree.
\item Object sizes are a prime number of cache lines to avoid
  associativity conflicts.
\end{itemize}

\end{frame}

\begin{frame}[fragile]
\frametitle{Wishlist}

\begin{itemize}
\item "MADV_FREE": cheaply decommits memory.  BSD has it.
Or an event indicating memory pressure.
\item Lock-aware scheduling: hint the kernel not to preempt (to
reduce lock-holder preemption).  Solaris has it.
\item Subscribable mutexes:  Need to sleep until a mutex unlocks, without locking it.
\item Late lock subscription: Can we do it safely?
\item Compiler-help for cache preloading.
%\item Why can't I get rid of those last 0.03\% of the lock fallbacks?
%\item Measure SuperMalloc in real workloads.  (Or at least some of the
%  common malloc benchmarks.)
\end{itemize}

\end{frame}

\begin{frame}
\frametitle{SuperMalloc is Available}

From {\small \url{https://github.mit.edu/SuperTech/SuperMalloc}},

or downstream at {\small \url{https://github.com/kuszmaul/SuperMalloc}}.

Dual licensed under MIT Expat and GPLv3.
\end{frame}

\end{document}
\punt{
\begin{frame}
\frametitle{Doug Lea's Goals For Malloc}

\begin{columns}
\hspace*{-.3cm}
\begin{column}{1.15 \textwidth}
\begin{description}[Error Detection:]
\item[Compatibility:] POSIX API.
\item[Portability:] SuperMalloc now works only on x86-64/Linux (and likes Haswell).
\item[Space:] SuperMalloc wins.
\item[Time:] SuperMalloc wins on average time.  Difficult to measure worst case.
\item[Tunability:] I hate tunability.
\item[Locality:] Objects allocated at the same time should be near each other.  Nobody seems to care.
\item[Error Detection:] SuperMalloc performs little checking.
\item[Anomalies:] I'm hopeful.
\end{description}
\end{column}
\end{columns}
\end{frame}
}


\begin{frame}
\frametitle{Worst Case Time is Bad}

My motivation: JEmalloc seems to be the best allocator right now.  It
is much faster than DLmalloc, and its memory footprint is half for
long-lived processes (such as database servers).

\vfill
However: In JEmalloc, once per day "free()" takes 3 seconds.

\vfill
I suspect lock-holder preemption, but it's tough to observe.
\end{frame}



\begin{frame}
\frametitle{DLmalloc malloc()}

\begin{itemize}
\item Find any object in the smallest nonempty bin that is big enough.

\item If none available, get  more memory from operating system.

\item Historically: Earlier versions of DLmalloc implemented first-fit within each bin.

They kept the bins sorted (but maintaining a heap in each bin would
have been enough).

\item Now it's more complex.

\item Operation complexity is $O(\mbox{\mintinline{c}{sizeof(size_t)}})$.
\end{itemize}
\end{frame}
  
\begin{frame}
\frametitle{DLmalloc free()}

\begin{enumerate}
\item Remove adjacent free blocks (if any) from their bins.
\item Coalesce adjacent free blocks.
\item Put the resulting free block in the right bin.
\end{enumerate}
\end{frame}

\begin{frame}
\frametitle{DLmalloc suffers an 8-byte/object overhead}

Since each object is preceeded by an 8-byte size, there is a 100\%
overhead on 8-byte objects.

\end{frame}

\begin{frame}
\frametitle{DLmalloc suffers fragmentation}

\begin{itemize}
\item DLmalloc, in the past, implemented first-fit, but does not appear to do so now.

\item DLmalloc maintains ``bins'' of objects of particular size ranges.

\item Small objects end up next to large objects.

\item Pages can seldom be returned to the operating system.

\item Compared to Hoard or JEmalloc, DLmalloc results in twice the
  resident set size (RSS) for long-lived applications, such as
  servers.
\end{itemize}

\end{frame}




\begin{frame}
\frametitle{Returning Pages to the Operating system}

\begin{itemize}
\item Empty pages can be released using
\mintinline{c}{madvise(p, MADV_DONTNEED, ...)}, which zeros the page while keeping the virtual address valid.

\item JEmalloc includes much complexity to overcome the high cost of
  \mintinline{c}{DONTNEED}.

\item SuperMalloc may not suffer as much.
\end{itemize}
\end{frame}

\begin{frame}
\frametitle{Performance of returning memory}

\begin{itemize}
\item On linux, avoid calling \mintinline{c}{munmap()}, which pokes
  holes in the virtual address space.  When many holes exist, Linux is slow to
  find a free range.

\item BSD offers \mintinline{c}{MADV_FREE}, which
  gives the kernel \textit{permission} to free memory, without \textit{requiring} it. The memory retains its old value, but can be zero'd at any time.
  The OS frees physical memory asynchronously, only when there is memory pressure, and often avoids the cost of
  deallocating/reallocation.

\item Kernel should deliver an event to the application when memory is tight.

\item Don't yet know if this is important for SuperMalloc.
\end{itemize}
\end{frame}

\begin{frame}
\frametitle{Large Objects in JEmalloc}
\begin{itemize}
\item For objects $>$ 4MiB, round up to a 4MiB boundary.
\item If you \mintinline{c}{malloc(1+(1<<22))} (slightly more than 4MiB), the system allocates 8MiB.  
\item This allocation uses up virtual space, but not RSS\:.
  The OS \textit{commits} physical memory for a
  page only when the application reads or writes the page.  Since the page
  size is 4KiB, in this example, at most 4MiB+4KiB of RSS is allocated.
\end{itemize}
\end{frame}
\begin{frame}
\frametitle{SuperMalloc Strategies}

\begin{itemize}
\item Can waste virtual space: it's $2^{48}$ bytes.
\item Don't waste RSS ($2^{40}$ bytes on big machines).
\item Contention costs dearly, not locking.  Use a per-CPU cache.
\item Make per-CPU cache smaller than L3 cache, since the application
  has cache misses anyway.
\item Thread cache should be just big enough to reduce locking overhead.  (About 10 objects.)
\item Use Hardware transactional memory (HTM).
\item HTM likes simple data structures.
\item Object sizes should be a prime number of cache lines to avoid associativity conflicts.
\end{itemize}
\end{frame}



%% \begin{frame}
%% \frametitle{Supermalloc Strategies}

%% \begin{itemize}
%% \item Virtual address space is $2^{48}$ bytes.  Real memory is typically $2^{33}$ up to $2^{41}$ bytes. 

%% Don't waste real memory.  Do waste virtual memory.

%% \item Locking uncontended locks is far cheaper than accessing contended cache lines.

%% Use a per-CPU cache.

%% \item Applications access their objects, incurring cache misses.

%% Set the per-CPU cache to be smaller than L3.

%% \item Thread cache reduces locking costs.

%% Thread cache holds only about 10 objects.

%% \item HTM likes simple data structures.

%% Use arrays instead of red-black trees.

%% \item Associativity conflicts cause trouble.

%% Object sizes should be a prime number of cache lines.

%% \end{itemize}
%% \end{frame}






\begin{frame}[fragile]
\frametitle{Per-Thread PRNG With No Initialization}

\begin{minted}[mathescape]{c}
// Mix64 is a hash function
static uint64_t Mix64 (uint64_t);

// per-thread pseudorandom number generator.
uint64_t prandnum() {
  static __thread uint64_t rv = 0;
  return Mix64(++rv + (uint64_t)&rv);
}
\end{minted}

(Due to Dave Dice:
\url{https://blogs.oracle.com/dave/entry/a_simple_prng_idiom})

I'm skeptical of this.  Different threads may be too correlated.  Can
it be fixed up?

\end{frame}


\begin{frame}
\frametitle{Performance Comparison}
\hspace*{-.7cm}\includegraphics{new-malloc-test-1K-aggregated.pdf}

{\small 8 runs on i7-4770: 3.4GHz, 4 cores, 2 HT/core, no turboboost.}
\end{frame}



\end{document}

\begin{frame}[fragile]
\frametitle{An Implementation of Memalign}

Idea: allocate a little extra, and then return adjust the pointer to be aligned.

\begin{minted}[mathescape]{c}
void* memalign(size_t alignment, size_t s) {
  size_t p = (size_t)malloc(s+alignment-1);
  size_t a = (p+alignment-1)&~(alignment-1);
  return (void*)a;
}
\end{minted}

Bug: This implementation is wrong, because objects returned by
\mintinline{c}{memalign()} can be passed to \mintinline{c}{free()},
which requires that its argument is a block returned by
\mintinline{c}{malloc()}.


\end{frame}



the problem(s) with JEmalloc
 visible problems:
   1) occasional 3-second call to free() (I don't know if I've fixed this, but it seems likely.  It's tough to reproduce)
   2) large memory footprint (essentially 2x?)
   3) Cache-index unfriendly
 mechanisms
   1) lowest-address allocation (an interesting heuristic.  May not be a problem itself. The data structure to calculate this may be a problem?)
   2) many arenas (the thread cache is too big?)

also want to investigate
 programming with transactional memory (this is weak, since I don't really see a performance advantage)
   1) What is TM
   2) How does it work (cached watching)
   3) Transactions fail--> need a fallback
     a) Subscribe to a lock (explain subscribe)
     b) Issues with late subscription/early subscription
   4) Why do transactions fail?
     a) interrupts (in particular, time slice interupt --> no transactions longer than 10ms)
     b) cache capacity (the read set can be in L2, the write set must fit in L1 - note this means that writes should be delayed if possible)
     c) actual conflicts
     d) Other random failures (the failure codes are essentially useless, except for the ``user aborted'' code)
   5) Tricks
     a) don't enter the transaction until the lock is available
     b) subscribe to the lock late
     c) prefetch data before entering the transaction (prefetch for write on writeable data) - this optimization doesn't seem to matter much)
     d) after doing all this, locks are just as fast as transactions


batch move from thread cache to global cache
